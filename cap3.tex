%%%%%%%%%%%%%%%%%%%%%%%%%%%%%%%%%%%%%%%%%%%%%%%%%%%%%%%%%%%%%%%%%%%%%%%%%%%%%%%
% Chapter 3: Resultados
%%%%%%%%%%%%%%%%%%%%%%%%%%%%%%%%%%%%%%%%%%%%%%%%%%%%%%%%%%%%%%%%%%%%%%%%%%%%%%%

%++++++++++++++++++++++++++++++++++++++++++++++++++++++++++++++++++++++++++++++

Finalizada la etapa de desarrollo del Trabajo de Fin de Máster, se procede a describir la herramienta implementada.


La herramienta se ha denominado ghhell, abreviatura de 'GitHub Shell'. Se ha publicado en NPM[Enlace] para su fácil distribución e instalación.

Las funcionalidades implementadas en ghshell, se describen a continuación:

\begin{itemize}
    \item Autenticación con GitHub.
        		[ Imagen ]
    \item Listar organizaciones, asignaciones y repositorios de GitHub del usuario.
        		[ Imagen ]
    \item Automatizar la descarga de repositorios.
        		[ Imagen ]
    \item Automatizar la ejecución de scripts en los repositorios (TDD, creación de entorno, evaluación de código...).
        		[ Imagen ]
    \item Recopilar la información obtenida de la automatización de tareas y presentarla al usuario (PDF, HTML...).
        		[ Imagen ]
\end{itemize}


%---------------------------------------------------------------------------------
\section{Problemas encontrados y soluciones}
\label{3:sec:1}

Asincronía, etc.

%---------------------------------------------------------------------------------
\section{Perfil del usuario de ghshell}
\label{3:sec:2}

El uso de ghshell está especialmente dirigido a un determinado grupo de profesores: nos referimos al perfil de un profesor, principalmente docente en alguna rama de Ingeniería, con conocimientos avanzados en programación y en herramientas de control de versiones.

No obstante, ya que la curva de aprendizaje de ghshell no es excesiva y dado que el uso de las herramientas de control de versiones no se limita exclusivamente a repositorios de código fuente, se puede extender su uso para el resto de profesorado y usuarios con otros roles. Basta con tener claras unas nociones básicas de informática, junto con la lectura y asimilación previa de la documentación de la herramienta.
