%%%%%%%%%%%%%%%%%%%%%%%%%%%%%%%%%%%%%%%%%%%%%%%%%%%%%%%%%%%%%%%%%%%%%%%%%%%%%
% Chapter 4: Conclusiones y Trabajos Futuros 
%%%%%%%%%%%%%%%%%%%%%%%%%%%%%%%%%%%%%%%%%%%%%%%%%%%%%%%%%%%%%%%%%%%%%%%%%%%%%%%

%++++++++++++++++++++++++++++++++++++++++++++++++++++++++++++++++++++++++++++++

Este capítulo es obligatorio.
Toda memoria de Trabajo de Fin de Máster ha de incluir unas conclusiones y unas 
líneas de trabajo futuro 

Desde hace unos años hasta ahora, ha tenido lugar un enorme crecimiento de las herramientas de control de versiones. Se han convertido en una herramienta imprescindible en la metodologías de desarrollo del software y las instituciones de enseñanza saben que incorporarlas a sus sistemas educativos es clave para ofrecer un servicio puntero y de calidad.

Ésto es lo que se pretende con la herramienta obtenida tras la realización de este Trabajo de Fin de Máster: que sea posible su implantación dentro del marco académico de la Universidad de La Laguna, partiendo de la premisa de que, actualmente, el desarrollo de un proyecto software sin tener detrás un sistema de control de versiones, no es viable.


......



Nuevo paradigma: Sincronía/asincronía
Motivación de aprender algo nuevo

Futuro:
ampliar funcionalidades: push, issues, despliegues...
