%%%%%%%%%%%%%%%%%%%%%%%%%%%%%%%%%%%%%%%%%%%%%%%%%%%%%%%%%%%%%%%%%%%%%%%%%%%%%
% Chapter 4: Conclusiones y Trabajos Futuros 
%%%%%%%%%%%%%%%%%%%%%%%%%%%%%%%%%%%%%%%%%%%%%%%%%%%%%%%%%%%%%%%%%%%%%%%%%%%%%%%

%++++++++++++++++++++++++++++++++++++++++++++++++++++++++++++++++++++++++++++++

Desde hace unos años hasta ahora, ha tenido lugar un enorme crecimiento de las herramientas de control de versiones. Se han convertido en una herramienta imprescindible en la metodologías de desarrollo del software y las instituciones de enseñanza saben que incorporarlas a sus sistemas educativos es clave para ofrecer un servicio puntero y de calidad.
\bigskip

Ésto es lo que se pretende con la herramienta obtenida tras la realización de este Trabajo de Fin de Máster: que sea posible su implantación dentro del marco académico de la Universidad de La Laguna, partiendo de la premisa de que, actualmente, el desarrollo de un proyecto software sin tener detrás un sistema de control de versiones, no es viable.
\bigskip

La automatización de las tareas de clonado y ejecución de pruebas facilitaría al profesor, en primera instancia, la corrección de las prácticas y proyectos de los alumnos. El ahorro de tiempo de ejecutar estas tareas manualmente es considerable, teniendo en cuenta el número de prácticas que realiza cada alumno por asignatura. Esta enorme carga de trabajo del profesor puede ser aprovechada en otros ámbitos docentes.
\bigskip

Por otra parte, esta herramienta sienta las bases a posibles desarrollos futuros, ampliando las funcionalidades de la misma. Se ha desarrollado pensando en su posible escalabilidad y ya que cuenta con toda la estructura base creada (autentificación de usuarios, clonado, ejecución y reporte de resultados), se pueden añadir funcionalidades sin demasiado esfuerzo.
\newpage

Para concluir, podemos afirmar que los objetivos marcados al comienzo de este Trabajo de Fin de Máster han sido cumplidos y las principales líneas de desarrollo a continuar podrían ser las enumeradas a
continuación:  

\begin{itemize}
	\item Dotar de más funcionalidad de GitHub a la herramienta:
	\begin{itemize}
		\item Subir cambios a los repositorios (git push).
		\item Crear issues.
		\item Gestionar Pull Requests.
		\item Buscar repositorios.
		\item Gestión de permisos de usuarios a repositorios y organizaciones.
		\item Gestionar Classrooms.
	\end{itemize}
	\item Enriquecer el formato de la documentación generada.
	\item Realizar despliegues locales de aplicaciones web (como procesos hijos de la herramienta).
\end{itemize}

