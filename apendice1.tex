{\bfseries {\Huge A}}\label{Apendice1:A}
\bigskip
\bigskip

\begin{description}
  \item[\underline{AJAX}\label{apend1:ajax}]: acr\'onimo de \textit{Asynchronous JavaScript And XML} (JavaScript as\'{\i}ncrono y XML). Es una t\'ecnica de desarrollo web para crear aplicaciones 
  interactivas o RIA (\textit{Rich Internet Applications}). Estas aplicaciones se ejecutan en el cliente, es decir, en el navegador de los usuarios mientras se 
  mantiene la comunicaci\'on as\'{\i}ncrona con el servidor en segundo plano. De esta forma es posible realizar cambios sobre las p\'aginas sin necesidad de 
  recargarlas, mejorando la interactividad, velocidad y usabilidad en las aplicaciones.
  \bigskip
\end{description}

\begin{description}
  \item[\underline{API}\label{apend1:api}]: (\textit{Application Programming Interface} o Interfaz de Programaci\'on de Aplicaciones). Conjunto de funciones y procedimientos o m\'etodos que 
  ofrece cierta librer\'{\i}a para ser utilizados por otro software como una capa de abstracci\'on. 
  \bigskip
\end{description}

\bigskip
{\bfseries {\Huge C}}\label{Apendice1:C}
\bigskip
\bigskip

\begin{description}
   \item[\underline{CVS}\label{apend1:cvs}]: (\textit{Control Versioning System} o Sistema de Control de Versiones). Aplicaci\'on inform\'atica que implementa un sistema de control de 
  versiones: mantiene el registro de todo el trabajo y los cambios en los ficheros (c\'odigo fuente principalmente) que forman un proyecto y permite que distintos desarrolladores 
  (potencialmente situados a gran distancia) colaboren.
  \bigskip
\end{description}

\bigskip
{\bfseries {\Huge D}}\label{Apendice1:D}
\bigskip
\bigskip

\begin{description}
  \item[\underline{DSL}\label{apend1:dsl}]: (\textit{Domain Specific Language} o Lenguaje de Dominio Espec\'{\i}fico). En desarrollo de software, un lenguaje de dominio espec\'{\i}fico es 
  un lenguaje de programaci\'on o especificaci\'on de un lenguaje dedicado a resolver un problema en particular, representar un problema espec\'{\i}fico 
  y proveer una t\'ecnica para solucionar una situaci\'on particular. 
\end{description}

{\bfseries {\Huge G}}\label{Apendice1:G}
\bigskip
\bigskip

\begin{description}
  \item[\underline{Google Drive}\label{apend1:drive}]: servicio de alojamiento de archivos accesible desde su p\'agina web mediante ordenadores o mediante aplicaciones m\'oviles. Para m\'as 
  informaci\'on, visitar {\small \url{https://drive.google.com}}.
  \bigskip
\end{description}

\begin{description}
  \item[\underline{GitHub}\label{apend1:github}]: forja para alojar proyectos utilizando el Sistema de Control de Versiones {\bfseries Git}. Para m\'as informaci\'on, visitar {\small 
  \url{https://github.com}}.
  \bigskip
\end{description}

\bigskip
{\bfseries {\Huge H}}\label{Apendice1:H}
\bigskip
\bigskip

\begin{description}
  \item[\underline{HTML5}\label{apend1:html}]: (\textit{HyperText Markup Language}). Lenguaje de marcado para la elaboraci\'on de p\'aginas web. Es un est\'andar que sirve de referencia para la 
  elaboraci\'on de p\'aginas web definiendo una estructura b\'asica y un c\'odigo para la definici\'on del contenido de la misma.
  \bigskip
\end{description}

\bigskip
\newpage

{\bfseries {\Huge J}}\label{Apendice1:J}
\bigskip
\bigskip

\begin{description}
  \item[\underline{JavaScript}\label{apend1:js}]: lenguaje de programaci\'on interpretado. Se define como orientado a objetos, basado en prototipos, imperativo, d\'ebilmente tipado y 
  din\'amico. Se utiliza principalmente en su forma del lado del cliente (\textit{client-side}), implementado como parte de un navegador web permitiendo mejoras en la interfaz de usuario y p\'aginas 
web din\'amicas.
  \bigskip
\end{description}

\bigskip
{\bfseries {\Huge M}}\label{Apendice1:M}
\bigskip
\bigskip

\begin{description}
  \item[\underline{Metodologias \'agiles}\label{apend1:ma}]: conjunto de m\'etodos de ingenier\'{\i}a del software basados en el desarrollo iterativo e incremental, donde los requisitos y 
  soluciones evolucionan mediante la colaboraci\'on de grupos auto organizados y multidisciplinarios. Se caracterizan adem\'as por la minimizaci\'on de riesgos desarrollando software en
  iteraciones cortas de tiempo.
  \bigskip
\end{description}

\bigskip
{\bfseries {\Huge O}}\label{Apendice1:O}
\bigskip
\bigskip

\begin{description}
  \item[\underline{OAuth 2.0}\label{apend1:oauth}]: (\textit{Open Authorization}): protocolo abierto que permite la autorizaci\'on segura de una API de modo est\'andar y simple para 
  aplicaciones de escritorio, web y m\'oviles. Para m\'as informaci\'on, visitar {\small \url{http://oauth.net/2/}}
  \bigskip
\end{description}
\newpage


{\bfseries {\Huge T}}\label{Apendice1:T}
\bigskip
\bigskip

\begin{description}
  \item[\underline{TDD}\label{apend1:tdd}]: (\textit{Test-Driven Development} o Desarrollo Dirigido por Pruebas). Pr\'actica de programaci\'on que involucra otras dos pr\'acticas: escribir las 
  pruebas primero (\textit{Test First Development}) y Refactorizaci\'on de c\'odigo (\textit{Refactoring}).
  \bigskip
\end{description}

\bigskip
{\bfseries {\Huge W}}\label{Apendice1:W}
\bigskip
\bigskip

\begin{description}
  \item[\underline{Web sem\'antica}\label{apend1:web}]: idea de a\~{n}adir metadatos sem\'anticos y ontol\'ogicos a la World Wide Web. Esas informaciones adicionales, que describen el contenido, 
  el significado y la relaci\'on de los datos, se deben proporcionar de manera formal, para que sea posible evaluarlas autom\'aticamente por m\'aquinas de procesamiento. El objetivo es mejorar 
  Internet ampliando la interoperabilidad entre los sistemas inform\'aticos usando {\bfseries agentes inteligentes}, es decir, programas en las computadoras que buscan informaci\'on sin necesidad de 
  interacci\'on humana.
  \bigskip
\end{description}

\begin{description}
  \item[\underline{World Wide Web}\label{apend1:www}]: (WWW). Sistema de distribuci\'on de documentos de hipertexto o hipermedios interconectados y accesibles v\'{\i}a Internet. Con un navegador 
  web, un usuario visualiza sitios web compuestos de p\'aginas web que pueden contener texto, im\'agenes, v\'{\i}deos u otros contenidos multimedia, y navega a trav\'es de esas p\'aginas usando 
hiperenlaces.
  \bigskip
\end{description}